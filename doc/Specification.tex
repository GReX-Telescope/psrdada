\documentclass{scrreprt}
%\usepackage[T1]{fontenc}
%\usepackage[latin1]{inputenc}
\usepackage{psfig}

\makeatletter
\makeatother

\begin{document}

\title{PuMa-II Software Specification}

\author{W.H.B.\ van Straten, B.W.\ Stappers, and K.\ Ramesh}

\maketitle
\tableofcontents{}

\begin{abstract}
This document specifies the software element of the 2nd generation of
the Dutch Pulsar Machine, PuMa-II, to be installed at the Westerbork
Synthesis Radio Telescope, WSRT, in 2005.  PuMa-II will consist of
eight data acquisition machines, called the \emph{primary units}, and
a processing cluster of \emph{secondary units}.

Each primary unit will interface with one of the 20\, MHz, 12-bit
sampled outputs of the WSRT tied-array beam former.  On each machine,
a custom-designed interface card (PiC) will re-sample the data with
8-bit resolution and write the data to computer RAM via a direct
memory access (DMA) interface.  A large file storage facility,
implemented using hardware RAID controllers, will be used to store
data locally, and a high-speed interconnect will enable the data to be
sent to secondary nodes for data reduction.

The communications plan and protocol used for sending the data from
primary to secondary node may change over the life of the PuMa-II
instrument.  Therefore, an adaptable and modular design philosophy has
been employed in software at every level of communications, command,
and control.  This document contains an overview of the software
model, including a discussion of the design decisions made, and
followed by a description of the requirements of each element.

\end{abstract}


\chapter{Introduction}

\section{Design Goals}

\begin{enumerate}
\item modularity
\vspace{-2mm}
\item separation of data transport, control, and data reduction
\vspace{-2mm}
\item near real-time data reduction
\vspace{-2mm}
\item elegant scaling to slower data reduction speeds
\end{enumerate}

\section{Overview}

\subsection{Workstation Cluster}

The functionality of PuMa-II will be divided across multiple workstations.
These are divided into two main classes:

\begin{enumerate}

\item Primary Units - workstations equipped with Direct Memory Access 
	(DMA) card, large data storage, and high-speed interconnect.

\item Secondary Units - workstations equipped with high-speed interconnect
	and modest data storage facilities.

\end{enumerate}

A single Primary unit may have multiple Secondary units associated
with it.  To these, it sends the digitized data in round-robin
fashion.  The various components of the workstation cluster are
depicted in simplified form in Figure~\ref{fig:layout}.

\begin{figure}
\centerline{\psfig{figure=layout.eps,width=4in,angle=0}}
\caption [\sffamily PuMa-II Schematic Overview]
{
Schematic Overview of PuMa-II.  The fat line indicates the direction
of data flow.  Note the parallel use of the RAM Data Block on both
Primary and Secondary Units.
}
\label{fig:layout}
\end{figure}

\subsection{Software}

The major functionality of PuMa-II is divided into five categories:

\begin{enumerate}

\item Data Flow Control - the high-bandwidth data transfer control software
\item Data Reduction - process and archive the baseband data
\item Command and Monitoring - communication channels for external control
\item Control Interface - centralized access to Command and Monitoring
\item Configuration and Scheduling - files and databases for automated control

\end{enumerate}

Communication between each of the levels of software will take place
primarily through shared memory resources and internet socket
connections.

\subsubsection{Data Flow Control}

Data Flow Control software will establish the high-bandwidth network
connections between the Primary and Secondary units.  This software
will control all aspects pertaining to the flow of data from the DMA
card on the Primary unit to local storage and/or the remote RAM and/or
disk storage of the Secondary units.  In the case of online pulsar
processing, this communication will include the overlap required to
compensate for data reduction losses (owing to dispersion smearing,
filter rise times, etc.).  

On both Primary and Secondary units, data will be stored in a large
buffer established as shared memory, called the Data Block.  The
various tasks that must run in parallel will be implemented as unique
processes, as opposed to multiple threads of a single process.
Therefore, access to the Data Block will be controlled by an
inter-process locking method, such as a semaphore.

Rational: It is better to begin with multiple processes and
inter-process control in the early states of development because this
paradigm is more modular.  For example, the process that reads data
from the Data Block and writes it to local storage may be run on
either a primary or secondary node.  If data reduction can later be
performed in real-time, the disk writing client may be replaced by a
data processing client.

\subsubsection{Data Reduction}

On both Primary and Secondary units, one or more Data Clients may
attach to the Data Block and operate on the data.  The Data Block will
be logically divided into a number of sub-blocks.  Each sub-block may
be flagged as ``processed'' by the clients in order that the Data Flow
Control software may over-write the information contained by that
sub-block.  After writing each sub-block, the Data Flow Control
software will set a flag to indicate that the block contains new data.

A single, high-priority Data Client will be given permission to flag
sub-blocks as completed.  Initially, this client will be part of the
Data Flow Control software that writes the data to local file storage.
Later, this client may be a data processing client.  Data Clients may
perform any number of tasks, including various forms of data
reduction, calculation and display of data quality statistics such as
the bandpass, storage of the data to some form of medium, or farming
the data out to a grid.

The data reduction client will basically do as it is told, as
described in Configuration and Scheduling.

\subsubsection{Command and Monitoring}

Command and Monitoring software includes the Command software that
establishes low-bandwidth network communication channels between
Primary and Secondary units and the Control Interfaces.  These
channels are used for sending high-level control commands, such as
start, stop, and information about the source.  These communication
channels may be implemented as a control thread in each component of
the Data Flow Control software. 

The Monitoring software will perform any tasks required to maintain
proper operation of the instrument and present useful information to
the user.  This includes monitoring telescope status information from
TCS, disk space consumption, network traffic, CPU load, etc.

\subsubsection{Control Interface}

The Control Interface software defines the centralized command/control
point, and will be connected to the various communication channels
established by the Control and Monitoring software on each of the
Primary and Secondary units.  The Control Interface may be run on any
workstation, and provides the means through which other processes may
treat PuMa-II as a single instrument.  For example, a text or
graphical user interface and/or automated scheduling program may
connect to PuMa-II, issue commands, and inquire about the status of
the instrument.  A textual user interface (TUI) will be developed to
connect to the Control Interface and:
\begin{itemize}
\item allow PuMa-II to be configured, started, and stopped;
\vspace{-2mm}
\item display various status variables; and
\vspace{-2mm}
\item create plots of diagnostic graphs, 
      such as the passband and digitizer statistics
\end{itemize}

\subsubsection{Configuration and Scheduling}

Configuration and scheduling will be make use of a database interface.
TO DO: Describe this in more detail.



\chapter{Primary Unit}

This machine talks directly to the PiC through its PCI interface (and
through the DMA card?).  It is responsible for farming out the
digitized data to multiple secondary units.

\section{Data Flow Control Software}

The Data Flow Control software running on the Primary and Secondary
Units is divided into three sections:
\begin{itemize}
\item Direct Memory Access (DMA)
\item File Input/Output (FIO)
\item Network Input/Output (NIO)
\end{itemize}

\subsection{Direct Memory Access (DMA)}

Direct Memory Access (DMA) transfer of digitized data from the PiC to
Primary Unit RAM occurs via a DMA card that is commercially available
from Engineering Design Team (EDT).  The DMA control software will:

\begin{enumerate}

\item allocate a number of fixed memory buffers of a size and number
to be determined during the testing stage;

\item send start and stop instructions to the PiC via the PCI/DMA interface;

\item copy filled memory buffers to the Data Block; and

\item monitor the number of buffers filled and copied, ensuring that
no data overflow occurs.

\end{enumerate}

\noindent
The DMA buffers will be separate from the Data Block buffers and
accessed only by the DMA card driver and a single process.  Once
started, DMA transfer will continue uninterrupted until a stop flag is
raised or an overflow occurs.

The Data Block will be allocated as a shared memory resource to which
only the DMA control process may write.  The Data Block will be
readable to any number of data monitoring processes, such as the
\emph{bandpass monitor}.  Only one process will be given the authority
to flag Data Block buffers as finished.

\subsection{File Input/Output (FIO)}

This process will read buffers from the Data Block, write them to
disk, and flag the written buffers as finished.  It can be run on
either Primary or Secondary Units, depending on the mode of operation.

\subsection{Network Input/Output (NIO)}

The software for network I/O will run on both Primary and Secondary
Units.  The protocol for the network communications will be a simple,
custom-built design on top of internet sockets.  This may change in
the future to some sort of grid-based protocol.

Header information (including all available observation information as
well as offset byte counts) will be sent with each block of data sent
from the Primary to Secondary Units.  The process running on each
Secondary unit will receive the data and write it to the Data Block in
its own shared memory.

The Secondary NIO software has the responsibility to monitor the
shared memory and ensure that there is sufficient space to hold
incoming data packets.  If there is insufficient space, the Primary
NIO should cease transfer to the current Secondary Unit and begin
transferring data to the next in the ring.


\chapter{Secondary Unit}

The Secondary units will contain a large amount of RAM and some
moderate data storage medium, such as a magnetic tape device or hard
drive.  Secondary units will receive large blocks of digitized data
from the Primary Unit via high-speed interconnect, and process the
data within a flexible system of handshaking.  Although any number of
Data Client processes may be run on the Secondary units, this paper
will provide details for two specific clients: a Data Storage Client,
and a Data Reduction Client.

\section{Data Flow Control Software}

Data Flow Control software running on the Secondary units will
establish and maintain high-bandwidth data communication channels with
a single Primary Unit.  Data received via this communication channel
will be copied to the Data Block.  By a handshaking protocol described
in Section~\ref{sec:data_block}, the Data Flow Control Software will
cooperate with Data Client Software and ensure that all received data
is processed (storage or reduction).

\section{Data Client Software}

Data Client software running on the Secondary units will read the
digitized data from the Data Block shared memory and operate on this
data.  Upon completion of a sub-block of data, it may be flagged as
processed, allowing the Data Flow Control software to write over this
sub-block.  This handshaking protocol is similar to that used by the
EDT DMA driver.

\subsection{Data Storage Client: {\tt db2disk}}

Write a block to disk, flag as processed.

\subsection{Data Reduction Client: {\tt dspsr}}

Process a block of data and flag as processed or process a file on
disk and remove it.

\section{Data Block}
\label{sec:data_block}

The Data Block will be allocated as a shared memory resource.  It will
consist of a primary header followed by a number of sub-blocks.  Each
sub-block will have a corresponding variable to indicate the state of
the block: empty, data valid, or data processed.

All processes may read from the Data Block shared memory, but only the
Data Flow Control process may write to it.  In addition, only one Data
Client will be given the permission to set the ``data processed'' flag
of the sub-blocks.

Only the data from one observation can be held in the Data Block at
one time.  The relevant observation information (such as bandwidth,
centre frequency, source, start time, etc.) is stored in the primary
header space of the Data Block.  Each sub-block with ``data valid''
will also have an associated byte count that may be used to calculate
the time offset from the start of the observation.

If the Data Flow Control software cannot obtain a ``data processed''
or ``empty'' sub-block, an overflow condition occurs, and overflow
handling routines must propagate an appropriate message to the Primary
unit.  Depending upon the mode of operation, this condition may be
interpreted as an error, or as a signal to move on to the next
Secondary unit in the data transmission queue.


\chapter{Summary}

The following table summarizes the software that will be developed,
its basic functionality, and the machine on which it will run.

\vspace{5mm}

\begin{tabular}{l|p{8cm}|l}

Name & \multicolumn{1}{c}{Function} & Machine \\ \hline

{\tt dma2db} & Transfers data from EDT buffers to the Data Block.
	& Primary \\

{\tt db2disk} & Reads data from the Data Block and writes it, with
	header information, to disk. & Both \\

{\tt db2nic} & Reads data from the Data Block and sends it, with
	header information, to a Secondary node. & Primary \\

{\tt nic2db} & Receives data from Primary node and writes it to the 
	Data Block. & Secondary \\

{\tt dspsr} & Attaches to the Data Block and processes raw data
	according to specification, writing results to disk. &
	Secondary \\

{\tt puma2} & Connects to the Command and Control interface of the
	various Primary nodes.  Accepts and maintains external
	text-based control connection.  & fixed \\

{\tt puma2tcs} & Translates TCS control structure packets into
	text-based commands as accepted by {\tt puma2} control
	connection. & fixed \\

{\tt puma2tui} & Textual user interface connects to {\tt puma2},
	displays various quantities of interest, and controls
	the instrument. & variable \\

{\tt puma2sched} & High-level code that schedules and records
        data reduction operations & fixed \\

\end{tabular}



\end{document}
