\documentclass{scrreprt}
\usepackage{psfig}
\usepackage{atbeginend}

\makeatletter
\makeatother

\BeforeBegin{itemize}{\setlength{\topsep}{-2mm}}
\BeforeBegin{itemize}{\setlength{\partopsep}{-2mm}}

\BeforeBegin{enumerate}{\setlength{\topsep}{-2mm}}
\BeforeBegin{enumerate}{\setlength{\partopsep}{-2mm}}

\AfterBegin{itemize}{\setlength{\itemsep}{-1mm}}
\AfterBegin{enumerate}{\setlength{\itemsep}{-1mm}}

\begin{document}

\title{PuMa-II Software Specification}

\author{W.H.B.\ van Straten, B.W.\ Stappers, and K.\ Ramesh}

\maketitle
\tableofcontents{}

\begin{abstract}
This document specifies the software element of the 2nd generation of
the Dutch Pulsar Machine, PuMa-II, to be installed at the Westerbork
Synthesis Radio Telescope, WSRT, in 2005.  PuMa-II will consist of
eight data acquisition machines, called the \emph{Primary} nodes, and
a processing cluster of \emph{Secondary} nodes.

Each Primary Node will interface with one of the 20\, MHz, 12-bit
sampled outputs of the WSRT tied-array beam former.  On each machine,
a custom-designed interface card (PiC) will re-sample the data with
8-bit resolution and write the data to computer RAM via a direct
memory access (DMA) interface.  A large file storage facility,
implemented as internal RAID, will be used to store data locally, and
a high-speed interconnect will be used to send the data to Secondary
nodes for data reduction.

The communications plan and protocol used for sending the data from
Primary to Secondary node may change over the life of the PuMa-II
instrument.  Therefore, an adaptable and modular design philosophy has
been employed in the data flow control software, based on the use of a
ring buffer in shared memory.  This document contains an overview of
the software model, including a discussion of the design decisions
made at each level of command, control, and configuration.

\end{abstract}


\chapter{Introduction}

\section{Design Goals}

\begin{enumerate}
\item modularity
\vspace{-2mm}
\item separation of data transport, control, and data reduction
\vspace{-2mm}
\item near real-time data reduction
\vspace{-2mm}
\item elegant scaling to slower data reduction speeds
\end{enumerate}

\section{Overview}

\subsection{Workstation Cluster}

The functionality of PuMa-II will be divided across multiple workstations.
These are divided into two main classes:

\begin{enumerate}

\item Primary Units - workstations equipped with Direct Memory Access 
	(DMA) card, large data storage, and high-speed interconnect.

\item Secondary Units - workstations equipped with high-speed interconnect
	and modest data storage facilities.

\end{enumerate}

A single Primary unit may have multiple Secondary units associated
with it.  To these, it sends the digitized data in round-robin
fashion.  The various components of the workstation cluster are
depicted in simplified form in Figure~\ref{fig:layout}.

\begin{figure}
\centerline{\psfig{figure=layout.eps,width=4in,angle=0}}
\caption [\sffamily PuMa-II Schematic Overview]
{
Schematic Overview of PuMa-II.  The fat line indicates the direction
of data flow.  Note the parallel use of the RAM Data Block on both
Primary and Secondary Units.
}
\label{fig:layout}
\end{figure}

\subsection{Software}

The major functionality of PuMa-II is divided into five categories:

\begin{enumerate}

\item Data Flow Control - the high-bandwidth data transfer control software
\item Data Reduction - process and archive the baseband data
\item Command and Monitoring - communication channels for external control
\item Control Interface - centralized access to Command and Monitoring
\item Configuration and Scheduling - files and databases for automated control

\end{enumerate}

Communication between each of the levels of software will take place
primarily through shared memory resources and internet socket
connections.

\subsubsection{Data Flow Control}

Data Flow Control software will establish the high-bandwidth network
connections between the Primary and Secondary units.  This software
will control all aspects pertaining to the flow of data from the DMA
card on the Primary unit to local storage and/or the remote RAM and/or
disk storage of the Secondary units.  In the case of online pulsar
processing, this communication will include the overlap required to
compensate for data reduction losses (owing to dispersion smearing,
filter rise times, etc.).  

On both Primary and Secondary units, data will be stored in a large
buffer established as shared memory, called the Data Block.  The
various tasks that must run in parallel will be implemented as unique
processes, as opposed to multiple threads of a single process.
Therefore, access to the Data Block will be controlled by an
inter-process locking method, such as a semaphore.

Rational: It is better to begin with multiple processes and
inter-process control in the early states of development because this
paradigm is more modular.  For example, the process that reads data
from the Data Block and writes it to local storage may be run on
either a primary or secondary node.  If data reduction can later be
performed in real-time, the disk writing client may be replaced by a
data processing client.

\subsubsection{Data Reduction}

On both Primary and Secondary units, one or more Data Clients may
attach to the Data Block and operate on the data.  The Data Block will
be logically divided into a number of sub-blocks.  Each sub-block may
be flagged as ``processed'' by the clients in order that the Data Flow
Control software may over-write the information contained by that
sub-block.  After writing each sub-block, the Data Flow Control
software will set a flag to indicate that the block contains new data.

A single, high-priority Data Client will be given permission to flag
sub-blocks as completed.  Initially, this client will be part of the
Data Flow Control software that writes the data to local file storage.
Later, this client may be a data processing client.  Data Clients may
perform any number of tasks, including various forms of data
reduction, calculation and display of data quality statistics such as
the bandpass, storage of the data to some form of medium, or farming
the data out to a grid.

The data reduction client will basically do as it is told, as
described in Configuration and Scheduling.

\subsubsection{Command and Monitoring}

Command and Monitoring software includes the Command software that
establishes low-bandwidth network communication channels between
Primary and Secondary units and the Control Interfaces.  These
channels are used for sending high-level control commands, such as
start, stop, and information about the source.  These communication
channels may be implemented as a control thread in each component of
the Data Flow Control software. 

The Monitoring software will perform any tasks required to maintain
proper operation of the instrument and present useful information to
the user.  This includes monitoring telescope status information from
TCS, disk space consumption, network traffic, CPU load, etc.

\subsubsection{Control Interface}

The Control Interface software defines the centralized command/control
point, and will be connected to the various communication channels
established by the Control and Monitoring software on each of the
Primary and Secondary units.  The Control Interface may be run on any
workstation, and provides the means through which other processes may
treat PuMa-II as a single instrument.  For example, a text or
graphical user interface and/or automated scheduling program may
connect to PuMa-II, issue commands, and inquire about the status of
the instrument.  A textual user interface (TUI) will be developed to
connect to the Control Interface and:
\begin{itemize}
\item allow PuMa-II to be configured, started, and stopped;
\vspace{-2mm}
\item display various status variables; and
\vspace{-2mm}
\item create plots of diagnostic graphs, 
      such as the passband and digitizer statistics
\end{itemize}

\subsubsection{Configuration and Scheduling}

Configuration and scheduling will be make use of a database interface.
TO DO: Describe this in more detail.

\chapter{Data Flow Control Software}

Data Flow Control software running on the Primary and Secondary nodes
must handle the flow of data in a modular and extensible manner,
enabling future developments by replacement of a single component.
The required modularity is met by basing all data transfer on a single
ring buffer protocol, which will be known as the Data Block.

\section{Data Block}
\label{sec:data_block}

The Data Block will be allocated as a shared memory resource,
logically divided into a header block followed by a number of
sub-blocks.  At the beginning of an observation, the header block will
be initialized with the relevant observation information (such as
bandwidth, centre frequency, source, start time, etc.) and the ring
buffered cleared.  Data will be written to sub-blocks in sequential
order until the end of the observation, at which point an end-of-data
flag will be raised.  Therefore, only one contiguous stream of data
may be represented in the Data Block at any one time.

Each sub-block will have a corresponding variable to indicate the
state of the block: empty, or valid.  Each sub-block flagged as valid
will also have an associated byte count that may be used to calculate
the time offset from the start of the observation.

A single, high-priority process, called the Write Client, will be
given write access to the Data Block; only the Write Client can change
the state of a sub-block from empty to valid.  The Write Client will
not write data to the next sub-block until its state is empty, and
after filling a sub-block will change its state to valid.

One or more Read Clients may attach to the Data Block and read the
data from sub-blocks marked as valid.  However, only a single,
high-priority Read Client will be given permission to change the state
of a sub-block from valid to empty.  This process will access
sub-blocks in contiguous order.

If the Write Client cannot obtain an empty sub-block, an overflow
condition will occur, and overflow handling routines must propagate an
appropriate message to the Command and Control software.  Depending
upon the mode of operation, this condition may be interpreted as an
error, or as a signal to move on to the next Secondary node in the
data transmission queue.

\newpage
\section{Data Flow Write Clients}

Write Client software will read data from a device and write it to
the Data Block.

\subsection{DMA Client: {\tt dma2db}}

The DMA Client software, {\tt dma2db}, is responsible for transferring
data from the telescope to the Data Block.  It will talk directly to
the PiC through its PCI interface, start and stop the data transfer,
and record the UTC start time of the observation.  Data from the PiC
will be transferred to Primary node RAM via a Direct Memory Access
(DMA) card that is commercially available from Engineering Design Team
(EDT).  The DMA Client software will:

\begin{enumerate}

\item allocate a number of fixed memory buffers of a size and number
to be determined during the testing stage;

\item send start and stop instructions to the PiC via the PCI/DMA interface;

\item determine the UTC start time of the first sample recorded

\item copy filled DMA buffers to the Data Block; and

\item monitor the number of DMA buffers filled and copied, ensuring that
no data overflow occurs.

\end{enumerate}

\noindent
The DMA buffers will be separate from the Data Block buffers and
accessed only by the DMA card driver and {\tt dma2db}.  Once started,
DMA transfer will continue uninterrupted until a stop flag is raised
or an overflow occurs.

\subsection{Network Interface Client: {\tt nic2db}}

The software for network I/O will run on both Primary and Secondary
nodes.  Data Flow Control software running on the Secondary nodes will
establish and maintain a high-bandwidth data communication channel
with a single Primary node.  The protocol for the network
communications will be a simple, custom-built design on top of
internet sockets.  This may change in the future to some sort of
grid-based protocol.  Data received via this communication channel
will be copied to the Data Block in contiguous order.  Each packet of
data will be preceded by a copy of the Data Block header from the
Primary node.  This header will be copied to the Secondary node Data
Block.

The {\tt nic2db} software has the responsibility to monitor the Data
Block and ensure that there is sufficient space to hold incoming data
packets.  It will send a message to the Primary node if there is
insufficient space, and the Primary node will cease data transfer,
possibly initiating data transfer to the next in Secondary node in the
ring.

\newpage
\section{Data Flow Read Clients}

Read Client software will read data from the Data Block and write it
to a device.

\subsection{Data Storage Client: {\tt db2disk}}

Writes data blocks to disk, breaking up data into files of arbitrary
length.  Each file will be preceded by the header block from the Data
Block.  Runs on either Primary or Secondary nodes, depending on the
mode of operation.

\subsection{Network Interface Client: {\tt db2nic}}

This software runs on the Primary nodes; it reads from the Data Block
and writes to one or more Secondary nodes, breaking up data into
packets of arbitrary length.  The total length of data sent to an
individual Secondary node will be independent of the Data Block buffer
sizes, and may depend on the overlap specified by the Configuration
and Scheduling software.

Header information (including all available observation information as
well as offset byte counts) will be sent with each packet of data sent
to the Secondary nodes.

\chapter{Data Reduction Software}

\section{Operational Phases}

The data reduction software will evolve through three phases in the life
of the PuMa-II instrument.

\begin{itemize}
\item {\bf offline} processing: performed after the recording has finished
\item {\bf simultaneous} processing: performed during the recording
\item {\bf real-time} processing: as above, but completely in memory
\end{itemize}

\subsubsection{Offline Data Reduction}

In the initial version of PuMa-II, the data reduction software will
operate only on data files stored on local disk space, and only after
the recording has completed.  After the observations have been
completed, the data files from each Primary node will be farmed out to
the Secondary nodes, and {\tt dspsr} will be run on each data file, as
specified by the Configuration and Scheduling software.

\subsubsection{Simultaneous Data Reduction}

In the next stage of development, the data reduction software will be
run during the observation.  In this mode, data are sent directly to
Secondary nodes and written to disk.  As before, {\tt dspsr} will be
run on the data files.

\subsubsection{Real-time Data Reduction}

In the final stage of PuMa-II evolution, the data reduction software
will be able to keep up completely with the flow of data on the
Primary nodes.  In this case, the data is never written to disk, and
{\tt dspsr} will operate as a Read Client directly connected to the
Data Block.  Another Read Client may be written that will farm data
reduction tasks to a grid computing facility.

\section{Configuration and Scheduling}

The data processing parameters will be specified by high-level
configuration and scheduling software.

\chapter{Command and Monitoring Software}

\section{Command}

Each element of the PuMa-II software must be coordinated to operate as
a single instrument.  Therefore, many of the processes described in
the previous chapter will have to be synchronized and configured
through communications channels.  Some degree of synchronization will
be achieved through the hand-shaking protocol of the Data Block
specification.  Other communication requirements will be met through
internet socket connections.

\subsection{Data Block Communications}

Many of the Data Clients can be implemented as a persistent process,
like a daemon, that is configured once during an initialization stage
and runs automatically from that point onward.  Apart from
configuration, the behaviour of these automatic processes will depend
completely upon the state of the Data Block.

Two Read Client programs that can run in this manner are {\tt db2disk}
and {\tt db2nic}.  These automatic processes need only start reading
from the Data Block when it is active, as determined by the behaviour
of the Write Client.  They start when the header is initialized and
stop when the end of data flag is raised.

Also, the {\tt nic2db} Write Client can be run as an automatic process
that starts when it receives packets from the Primary Node and ends
when the end of data message is received.

If it is shown that the operation of these Data Clients may have to
change from observation to observation, then there are two possible
solutions:
\begin{itemize}
\item Stop and restart the daemons with different configuration parameters
\item Enable socket communications that set configuration parameters
\end{itemize}

\subsection{Internet Socket Communications}

Certain processes will require internet socket communications in order
to be configured between observations and to start and stop the
observation.  In order that communications may be sent and received
during normal operation, the processes that require socket
communications will be multi-threaded.  The communication threads may
have lower priority than the main thread, if required.  More than one
communication channel may be opened to each process; however, only one
channel will be able to issue control commands.  The others may only
inquire about the status of the process.

All communications will be human readable, ASCII text.  This enables
interface testing using standard tools like telnet.  If large amounts
of binary data must be sent between the Control Interface and Data
Flow Control software, then it should be done using a separate
communication channel designated for this purpose.  Text commands will
be sent on a single line of text.  After every command received, the
process will respond with {\tt ok} or {\tt fail}, followed by any
additional information, and ending with the command prompt.

\subsubsection{DMA Data Client}

The {\tt dma2db} processes running on each of the primary nodes
require careful synchronization with the Control Interface software,
especially if they are all to be started on the same UTC second.  For
this reason, it will not suffice to remotely start the processes at
the desired beginning of each observation.  The processes must be
persistent and must maintain socket communications with the Control
Interface.  The following commands will be supported over this
interface:

\begin{itemize}
\item {\tt START} start recording data at the next interval; continue
  until {\tt STOP} command is received
\item {\tt START seconds} start recording for the specified number of
  seconds
\item {\tt STOP} stop the current recording
\item {\tt SET key=value} set the specified Data Block header parameter
\item {\tt GET key}
\item {\tt OVERLAP bytes} set the amount by which output data blocks
  should overlap
\end{itemize}

The {\tt OVERLAP} parameter is set in the Data Block header, and is used
by {\tt db2nic} to determine the amount by which data sent to different
Secondary nodes should overlap.

\section{Monitoring}

Monitoring processes will be run on all nodes in the PuMa-II
instrument, reporting on remaining disk space, CPU load, network
traffic, etc.  At the time of this writing, it is not clear if
standard cluster monitoring tools like Ganglia will suffice.  For
example, it may prove useful to have a regular update of which
Secondary nodes are currently receiving data.  This information would
have to come from {\tt db2nic}, possibly via a socket connection to
this process.

In addition to live monitoring, it may also prove useful to maintain a
database of relevant statistics, such as the average time required to
write a block of data to file or over the network.  These monitoring
tasks would be performed by the relevant process, {\tt db2disk} and/or
{\tt db2nic} and communicated to a central database via some protocol.

\newpage
\section{Primary Write Client Command Interface}
\label{sec:pwc}

This section describes the behaviour of the Write Client software that
will run on each of the Primary nodes, known as the Primary Write
Client (PWC) software.  In the case of PuMa-II, the PWC is 
{\tt puma2\_dmadb}.

\subsection{Operational States}

The PWC has four main states of operation:
\vspace{-2mm}
\begin{itemize}
\item {\bf idle} waiting for configuration parameters
\vspace{-2mm}
\item {\bf prepared} configuration parameters received; waiting for start
\vspace{-2mm}
\item {\bf recording invalid} recording data in over-write mode
\vspace{-2mm}
\item {\bf recording valid} recording data in lock-step mode
\end{itemize}

\subsubsection{Idle State}

In the idle state, the PWC sleeps until configuration parameters are
sent from the control software.  All configuration paramters are sent
in a single ASCII header.  This header is copied to the Header Block,
and the PWC enters the {\bf prepared} state.

\subsubsection{Prepared State}

In the prepared state, the PWC sleeps until a start command is sent
from the control software.  There are three different start commands
that can be received in this state:

\begin{itemize}
\item {\tt\bf INV\_START} enter the {\bf recording invalid} state
\vspace{-2mm}
\item {\tt\bf START} enter the {\bf recording valid} state
\vspace{-2mm}
\item {\tt\bf START $\langle duration\rangle$} same as {\tt START}, record
	for the duraction specified in either {\it seconds}, {\it samples},
	or {\it HH:MM:SS}.
\end{itemize}
For each of the above commands, the PWC will enter the specified state
at the next available opportunity (for PuMa-II, on the next {\tt
SYSTICK}).

\subsubsection{Recording Invalid State}

In this state, the PWC software clocks data into the Data Block but
does not flag the data as valid.  The PWC will overwrite the data in
each sub-block, and will remain in this state until one of the
following commands is received:

\begin{itemize}
\item {\tt\bf STOP} enter the {\bf idle} state immediately
\vspace{-2mm}
\item {\tt\bf VAL\_START YYYY-MM-DD-hh:mm:ss} raise the valid data flag
	at the specified UTC time in the data stream and enter the {\bf
	recording valid} state
\vspace{-2mm}
\item {\tt\bf VAL\_START YYYY-MM-DD-hh:mm:ss $\langle duration\rangle$} same 
	as {\tt VAL\_START}, record for the duraction specified in either
	{\it seconds}, {\it samples}, or {\it HH:MM:SS}.
\end{itemize}

Note that the UTC time specified in the first argument to {\tt
VAL\_START} may be any time in the future.  If it is in the past, then
the difference between the specified UTC and the present cannot be
greater than the amount of time corresponding to the length of the
Data Block.

\subsubsection{Recording Valid State}

In this state, the PWC software clocks data into the Data Block, flags
the data as valid, and will not overwrite a sub-block until it has
been flagged as cleared.  The PWC will remain in this state until one
of the following commands is received:

\begin{itemize}
\item {\tt\bf STOP} enter the {\bf idle} state immediately
\vspace{-2mm}
\item {\tt\bf STOP YYYY-MM-DD-hh:mm:ss} enter the {\bf idle} state
	at the specified time
\vspace{-2mm}
\item {\tt\bf VAL\_STOP YYYY-MM-DD-hh:mm:ss} raise the end of data flag
	at the specified UTC time in the data stream and enter the {\bf
	recording invalid} state
\end{itemize}

Note that the UTC time specified in the first argument to both {\tt STOP}
and {\tt VAL\_STOP} {\bf must} be in the future.



\chapter{Control Interface Software}

The Control Interface software will implement the single instrument
look and feel of PuMa-II.  A single process, {\tt puma2}, will:

\begin{itemize}
\item connect to the Command and Monitoring interface of required processes
\item issue commands as necessary to the required processes
\item collate monitoring information from the cluster nodes
\item provide Command and Monitoring connections to outside world
\item start and stop PuMa-II processes on any node as required
\end{itemize}

The number and variety of Data Flow Control Command and Monitoring
connections required will depend upon the amount of flexibility
desired. For example, if a configuration parameter such as the size of
the data blocks written to Secondary node (or file) must change from
observation to observation, then it will be necessary to communicate
these changes to the {\tt db2nic} or {\tt db2disk} processes on each
Primary node.  It will also be necessary to associate an observation
ID with each configuration.  Because of the large file storage space
on the Primary nodes, the {\tt db2nic} process may significantly lag
(in time) the primary control process.  Therefore, the configuration
commands that it receives would have to be queued.  Even with only the
Data Block buffer, the Data Client process could be busy when
configuration commands for the next observation are sent.

In addition to maintaining the required control connections with
processes on each of the Primary nodes, the Control Interface software
will also present a Command and Monitoring connection to the outside
world.  To this connection, an operator can connect to issue commands
and check on the status of the instrument/observation.  As before,
multiple communication channels may be connected, but only one will be
able to send control commands.  The operator will connect using a
textual user interface program, {\tt puma2tui}, which will present all
information from the Control Interface in an organized manner,
possibly employing the curses terminal control library.

Another process, {\tt puma2tms}, will be written to provide automated
control of the PuMa-II instrument by the WSRT TMS.  Commands sent by
TMS will be received by {\tt puma2tms}, converted to text commands,
and sent to {\tt puma2} through the operator interface.

The Control Interface software will enable complete initialization of
the PuMa-II instrument through a single command.  That is, on any or
all nodes, it will be able to start and stop various processes, create
and destroy the Data Block shared memory and semaphore resources, and
perform whatever other tasks prove useful in the initialization and
configuration of the PuMa-II instrument.


\chapter{Configuration and Scheduling Software}

PuMa-II configuration will depend upon the objectives of the
experiment and may change from observation to observation.

\section{Configuration}

The configuration software will set up various operational parameters,
including:

\begin{itemize}
\item DMA buffer size
\item Data Block buffer size
\item network packet size
\item number of Primary nodes
\item number of Secondary nodes
\item assignment of Secondary to Primary node
\item operational mode: offline, simultaneous, or diskless
\item overlap required between Secondary nodes
\end{itemize}

\section{Scheduling}

The scheduling software will work with a database in order to decide
upon the best PuMa-II instrument configuration and data reduction
operations.  Before the {\bf diskless} mode of data reduction is
implemented, all data will exist as a file on either the Primary or
Secondary nodes.  Whenever a file is written to disk, an entry will be
registered in a centralized {\bf observations database}, which will
contain basic header information such as
\begin{itemize}
\item source name
\item start time (UTC)
\item centre frequency (MHz)
\item band width (MHz)
\end{itemize}
as well as the location (machine and file name) of the data.  Each
entry will also contain a time-stamped list of {\bf performed
operations}, describing when the data was written, when and how it was
processed, when it was deleted, etc.  The header information will be
used to find matches in a {\bf configuration database}, which will
specify the instrument configuration and data reduction requirements
for each source.  An observation may be processed in multiple ways, as
specified by the list of {\bf requested operations} in the
configuration database entry.

If it is possible to achieve two requested operations in one execution
of {\tt dspsr} then this will be done.  Otherwise, data reduction
operation will be performed one at a time.  After each operation is
completed, it will be recorded in the list of performed operations of
the observations database entry.

The scheduling software will periodically check or poll the
observations database.  Any entries that require data reduction will
be scheduled according to the data reduction parameters of the
requested operations in the corresponding configuration database
entry.  An observation will be considered completely processed when
the list of performed operations is equal to the list of requested
operations.  At this point, the raw data will be deleted or archived.

Entries in the configuration database may have an expiration date
associated with them.  In this manner, the observer may specify
special configuration and/or reduction options for a specific
experiment without permanently corrupting the default behaviour.


\chapter{Summary}

The following table summarizes the software that will be developed,
its basic functionality, and the machine on which it will run.

\vspace{5mm}

\begin{tabular}{l|p{8cm}|l}

Name & \multicolumn{1}{c}{Function} & Machine \\ \hline

{\tt dma2db} & Transfers data from EDT buffers to the Data Block.
	& Primary \\

{\tt db2disk} & Reads data from the Data Block and writes it, with
	header information, to disk. & Both \\

{\tt db2nic} & Reads data from the Data Block and sends it, with
	header information, to a Secondary node. & Primary \\

{\tt nic2db} & Receives data from Primary node and writes it to the 
	Data Block. & Secondary \\

{\tt dspsr} & Attaches to the Data Block and processes raw data
	according to specification, writing results to disk. &
	Secondary \\

{\tt puma2} & Connects to the Command and Control interface of the
	various Primary nodes.  Accepts and maintains external
	text-based control connection.  & fixed \\

{\tt puma2tcs} & Translates TCS control structure packets into
	text-based commands as accepted by {\tt puma2} control
	connection. & fixed \\

{\tt puma2tui} & Textual user interface connects to {\tt puma2},
	displays various quantities of interest, and controls
	the instrument. & variable \\

{\tt puma2sched} & High-level code that schedules and records
        data reduction operations & fixed \\

\end{tabular}



\appendix
\chapter{Data Block}

In this chapter, the Data Block API is specified in detail.  The Data
Block is the ring buffer through which the primary data flow will take
place on both primary and secondary nodes in the PuMa-II cluster.
Access to the ring buffer shared memory is controlled by an
inter-process communication semaphore.

The Data Block API includes software for creating and initializing the
shared memory and semaphore resources, locking the shared memory into
physical RAM, connecting to the ring buffer, writing data to the ring
buffer and reading data from the ring buffer.

\section{Creation, Connection, and Destruction}

The Data Block ring buffer is accessed through a data type named {\tt
ipcbuf\_t}, which is declared and initialized as in the following
example:
\begin{verbatim}
#include "ipcbuf.h"
ipcbuf_t ringbuf = IPCBUF_INIT;
\end{verbatim}
To create a ring buffer, call
\begin{verbatim}
int ipcbuf_create (ipcbuf_t* ptr, int key, uint64 nbufs, uint64 bufsz);
\end{verbatim}
\vspace{-6mm}
\begin{itemize}
\item {\tt ptr} is a pointer to an unallocated ring buffer handle
\vspace{-2mm}
\item {\tt key} is a unique identifier (range of acceptable values???)
\vspace{-2mm}
\item {\tt nbufs} is the number of sub-blocks in the ring buffer
\vspace{-2mm}
\item {\tt bufsz} is the size of each sub-block in the ring buffer
\end{itemize}
After the ring buffer has been created, it is ready for use.   The ring
buffer resources will remain available until calling
\begin{verbatim}
int ipcbuf_destroy (ipcbuf_t* ptr);
\end{verbatim}
\vspace{-6mm}
\begin{itemize}
\item {\tt ptr} is a pointer to an allocated ring buffer handle
\end{itemize}
That is, even if the process that created the ring buffer exits, the
shared memory and semaphore resources will remain allocated in
computer memory.  In order to connect to a previously created Data
Block ring buffer, call
\begin{verbatim}
int ipcbuf_connect (ipcbuf_t* ptr, int key);
\end{verbatim}
\vspace{-6mm}
\begin{itemize}
\item {\tt ptr} is a pointer to an unallocated ring buffer handle
\vspace{-2mm}
\item {\tt key} is the unique identifier passed to {\tt ipcbuf\_create}
\end{itemize}
To disconnect, call
\begin{verbatim}
int ipcbuf_disconnect (ipcbuf_t* ptr);
\end{verbatim}
\vspace{-6mm}
\begin{itemize}
\item {\tt ptr} is a pointer to a connected ring buffer handle
\end{itemize}
Note that, after calling {\tt ipcbuf\_create}, the process is connected
to the newly-created ring buffer and it is not necessary to call 
{\tt ipcbuf\_connect}.  Similarly, after calling {\tt ipcbuf\_destroy},
it is not necessary (or possible) to call {\tt ipcbuf\_disconnect}.
After the process is connected to the Data Block ring buffer, it is
possible to write or read data.

\subsection{Locking into Physical RAM}

In order to ensure that the Data Block ring buffer remains in RAM and
is not swapped out by the virtual memory manager, call
\begin{verbatim}
int ipcbuf_lock_shm (ipcbuf_t* ptr);
\end{verbatim}
and, to unlock,
\begin{verbatim}
int ipcbuf_unlock_shm (ipcbuf_t* ptr);
\end{verbatim}


\section{Writing to the Data Block}

After connecting to the Data Block ring buffer, it is
possible to write data to it.

\subsection{Locking and Unlocking Write Access}

Naturally, only one process may write data to the ring buffer;
therefore, the Write Client must first lock write access to the buffer
by calling,
\begin{verbatim}
int ipcbuf_lock_write (ipcbuf_t* ptr);
\end{verbatim}
\vspace{-6mm}
\begin{itemize}
\item {\tt ptr} is a pointer to a connected ring buffer handle.
\end{itemize}
Similarly, write permission may be relinquished by calling
\begin{verbatim}
int ipcbuf_unlock_write (ipcbuf_t* ptr);
\end{verbatim}
\vspace{-6mm}
\begin{itemize}
\item {\tt ptr} is a pointer to a connected ring buffer handle.
\end{itemize}

\subsection{Write Loop}

After locking write access to the Data Block ring buffer, the Write
Client will generally enter a loop in which it
\begin{enumerate}
\item requests the next sub-block to which data may be written, 
\vspace{-2mm}
\item fills the sub-block
\vspace{-2mm}
\item marks the sub-block as filled
\end{enumerate}
Step 1 is performed by calling
\begin{verbatim}
char* ipcbuf_get_next_write (ipcbuf_t* ptr);
\end{verbatim}
\vspace{-6mm}
\begin{itemize}
\item {\tt ptr} is a pointer to a connected ring buffer handle
\vspace{-2mm}
\item RETURN value is the pointer to the next available sub-block
\end{itemize}
\begin{verbatim}
int ipcbuf_mark_filled (ipcbuf_t* ptr, uint64 nbytes);
\end{verbatim}
\vspace{-6mm}
\begin{itemize}
\item {\tt ptr} is a pointer to a connected ring buffer handle
\vspace{-2mm}
\item {\tt nbytes} is the number of valid bytes in the sub-block
\end{itemize}

If {\tt nbytes} is less than the number of bytes in each sub-block, as
set by the {\tt bufsz} argument to {\tt ipcbuf\_create}, then an
end-of-data condition is set.

\subsection{Writing before Start-of-Data}

By default, when a Data Block ring buffer is created, the
start-of-data state is enabled and any data written by the Write
Client will be made available to the Read Client.  However, in some
cases it may be useful for the Write Client to write data to the Data
Block before making it available to the Write Client.  For example,
the trigger to begin data acquisition may arrive later than the
desired data acquisition start time.

To begin writing data before the actual start of valid data, it is
necessary to first disable the start-of-data flag by calling
\begin{verbatim}
int ipcbuf_disable_sod (ipcbuf_t* ptr);
\end{verbatim}
\vspace{-6mm}
\begin{itemize}
\item {\tt ptr} is a pointer to a connected ring buffer handle
\end{itemize}
The Write Client may then enter a loop identical to that described in
the previous section: requesting, filling, and marking.  However, when
the start-of-data flag is disabled, the message that a sub-block has
been filled is not passed on to the Read Client and the Write Client
will over-write filled sub-blocks as necessary.  The Write Client
raises the start-of-data flag by calling
\begin{verbatim}
int ipcbuf_enable_sod (ipcbuf_t* ptr, uint64 st_buf, uint64 st_byte);
\end{verbatim}
\vspace{-6mm}
\begin{itemize}
\item {\tt ptr} is a pointer to a connected ring buffer handle
\vspace{-2mm}
\item {\tt st\_buf} is the absolute count of the first valid sub-block
\vspace{-2mm}
\item {\tt st\_byte} is the first valid byte in the first valid sub-block
\end{itemize}

Note that {\tt st\_buf} is an absolute sub-block count, equal to the
total number of sub-blocks filled before the first valid sub-block.
Naturally, it is not possible to raise the start-of-data flag for a
buffer that has already been over-written.  Therefore, the start
sub-block count plus the total number of sub-blocks must be greater
than the current sub-block count, or
\begin{verbatim}
st_buf > ipcbuf_get_write_count - ipcbuf_get_nbufs
\end{verbatim}


\section{Reading from the Data Block}

After connecting to the Data Block ring buffer, it is possible to read
data from it.  

\subsection{Locking and Unlocking Read Access}

Only one process may remove data from the ring buffer by flagging it
as cleared.  This process, the Read Client, must first lock read
access to the buffer by calling,
\begin{verbatim}
int ipcbuf_lock_read (ipcbuf_t* ptr);
\end{verbatim}
\vspace{-6mm}
\begin{itemize}
\item {\tt ptr} is a pointer to a connected ring buffer handle
\end{itemize}
Similarly, read permission may be relinquished by calling
\begin{verbatim}
int ipcbuf_unlock_read (ipcbuf_t* ptr);
\end{verbatim}
\vspace{-6mm}
\begin{itemize}
\item {\tt ptr} is a pointer to a connected ring buffer handle
\end{itemize}

\subsection{Read Loop}

After locking read access to the Data Block ring buffer, the Read Client
will generally enter a loop in which it
\begin{enumerate}
\item requests the next sub-block containing data, 
\vspace{-2mm}
\item operates on the data in the sub-block
\vspace{-2mm}
\item marks the sub-block as cleared
\end{enumerate}
Step 1 is performed by calling
\begin{verbatim}
char* ipcbuf_get_next_read (ipcbuf_t* ptr, uint64* bytes);
\end{verbatim}
\vspace{-6mm}
\begin{itemize}
\item {\tt ptr} is a pointer to a connected ring buffer handle
\vspace{-2mm}
\item {\tt bytes} will be set to the number of valid bytes in the sub-block
\vspace{-2mm}
\item RETURN value is the pointer to the first valid byte in the sub-block
\end{itemize}
Step 3 is performed by calling
\begin{verbatim}
int ipcbuf_mark_cleared (ipcbuf_t* ptr);
\end{verbatim}
\vspace{-6mm}
\begin{itemize}
\item {\tt ptr} is a pointer to a connected ring buffer handle
\end{itemize}

\input{header}
\chapter{Operational Logging}

The various messages produced by the data acquisition software must be
logged and/or communicated to possibly more than one listener.  Therefore,
all messages will be sent using the {\tt multilog} API.  A multilog session
is opened by calling
\begin{verbatim}
multilog_t* multilog_open (char syslog);
\end{verbatim}
\vspace{-6mm}
\begin{itemize}
\item {\tt syslog} if non-zero, all messages are cc'd to syslog
\end{itemize}
and closed by calling
\begin{verbatim}
int multilog_close (multilog_t* log);
\end{verbatim}
\vspace{-6mm}
\begin{itemize}
\item {\tt log} pointer to an open multilog session
\end{itemize}
Once opened, file streams may be added by calling
\begin{verbatim}
int multilog_add (multilog_t* log, FILE* fptr);
\end{verbatim}
\vspace{-6mm}
\begin{itemize}
\item {\tt log} pointer to an open multilog session
\vspace{-2mm}
\item {\tt fptr} pointer to an open file stream
\end{itemize}
Messages are written to all file streams (and syslog, if enabled) by calling
\begin{verbatim}
int multilog (multilog_t* log, int priority, const char* format, ...);
\end{verbatim}
\vspace{-6mm}
\begin{itemize}
\item {\tt log} pointer to an open multilog session
\vspace{-2mm}
\item {\tt priority} a {\tt syslog} priority
\vspace{-2mm}
\item {\tt format} an fprintf-style formatting string
\vspace{-2mm}
\item {\tt ...} the value(s) to be printed according to the format
\end{itemize}

Output messages are assigned a priority as described in the manpage
for the standard C {\tt syslog} utility.

\section{Example}

\begin{verbatim}
#include "multilog.h"

[...]

  /* open a connection to syslogd using the standard C facility */
  openlog ("dada_db2disk", LOG_CONS, LOG_USER);

  /* open a multilog session that will use syslog */
  multilog_t* log = multilog_open (1);

  /* copy all messages to the standard error */
  multilog_add (log, stderr);

  /* write a message */
  char* world_name = "Earth";
  int world_number = 1;

  multilog (log, LOG_INFO, "Hello %s %d", world_name, world_number);
\end{verbatim}

\chapter{Socket Communications}

\section{Examples}
\subsection{Example Server}

\begin{verbatim}
#include "sock.h"

[...]

  char hostname [100];
  int port = 20013;

  /* Ask for the fully qualified hostname ... */
  sock_getname (hostname, 100, 1);
  /* ... or, ask for the IP address */
  sock_getname (hostname, 100, 0);

  int sfd = sock_create (&port);
  if (sfd < 0)
    perror ("Error creating socket");

  fprintf (stderr, "listening on %s %d\n", hostname, port);
  int cfd = sock_accept (sfd);
  if (cfd < 0)
    perror ("Error accepting connection");
\end{verbatim}
The open file descriptor returned by {\tt sock\_accept} may be used
in calls to the standard C I/O routines, {\tt read} and {\tt write}
as well as {\tt send} and {\tt recv}.  Furthermore, the open file
descriptor can be converted into a stream by calling {\tt fdopen}.
It is important to note that a stream should be opened for only read
or write access, never both; e.g.
\begin{verbatim}
  /* two separate file streams are required for reading and writing */
  FILE* sockin = fdopen (cfd, "r");
  FILE* sockout = fdopen (cfd, "w");

  /* set the socket output to be line-buffered */
  setvbuf (sockout, 0, _IOLBF, 0);
\end{verbatim}
The server can now read from {\tt sockin} using standard C stream I/O
routines such as {\tt fscanf}, {\tt fread}, and {\tt fgets}.  It can
also write to {\tt sockout} using {\tt fprintf} and {\tt fwrite}.

\subsection{Example Client}

\begin{verbatim}
#include "sock.h"

[...]

  char* hostname = "puma2.tms.nl";
  int port = 20013;

  /* Connect to the specified host and port */
  int cfd = sock_open (hostname, port);
  if (cfd < 0)
    perror ("Error opening socket");

  fprintf (stderr, "connected to %s %d\n", hostname, port);
\end{verbatim}
As with the server, the socket file descriptor may be accessed using
standard C stream I/O routines by calling {\tt fdopen}.

\input{dada_pwc}
\chapter{Primary Write Client Main Loop}
\label{app:dada_pwc_main}



\input{dada_prc_main}
\chapter{Testing the DADA Software}
\label{app:software}

This chapter describes the various programs that have been designed to
facilitate testing and development of the DADA software

\section{Primary Write Client Demonstration, {\tt dada\_pwc\_demo}}

The Primary Write Client (PWC) Demonstration program, {\tt
dada\_pwc\_demo}, implements an example PWC interface.  It does not
actually acquire any data and therefore can be run on any computer.
This program has two modes of operation: free and locked.

In {\em free} mode, {\tt dada\_pwc\_demo} does not connect to the
Header and Data Blocks; therefore, it is not necessary to create the
shared memory resources and run a Primary Read Client program.  This
mode is most useful when testing the command interface and state
machine of the Primary Write Client.  To run in {\em free} mode,
simply type
\begin{verbatim}
dada_pwc_demo
\end{verbatim}

In {\em locked} mode, {\tt dada\_pwc\_demo} connects to the Header and
Data Blocks; therefore, it is necessary to first create the shared
memory resources and also run a Primary Read Client program, such as
{\tt dada\_dbdisk}.  This mode is most useful when testing the
interface between Primary Write Client, Header and Data Blocks, and
Primary Read Client software.  To run in {\em locked} mode, for
example
\begin{verbatim}
dada_db -d         # destroy existing shared memory resources
dada_db            # create new shared memory resources
dada_pwc_demo -l   # run in locked mode
\end{verbatim}
The first step is particularly useful when debugging.
In another window on the same machine, you might also run
\begin{verbatim}
dada_dbdisk -WD /tmp
\end{verbatim}

To connect with the PWC demonstration program and begin issuing
commands, simply run
\begin{verbatim}
telnet localhost 56026
\end{verbatim}
or replace {\tt localhost} with the name of the machine on which the
program is running.

\subsection{Primary Write Client Command, {\tt dada\_pwc\_command}}

It is also possible to control one or more instance of {\tt
dada\_pwc\_demo} using the Primary Write Client Command program, {\tt
dada\_pwc\_command}.  By default, {\tt dada\_pwc\_command} uses the
same port number (56026) as {\tt dada\_pwc\_demo}.  Therefore, if you
are running both programs on the same machine, it will be necessary to
specify a different port number for the command interface; e.g.
\begin{verbatim}
dada_pwc_command -p 20013
\end{verbatim}

The PWC command program must be configured as described in ???;
it also requires the use of specification files to prepare for recording.


\end{document}
