\chapter{Testing the DADA Software}
\label{app:software}

This chapter describes the various programs that have been designed to
facilitate testing and development of the DADA software

\section{Primary Write Client Demonstration, {\tt dada\_pwc\_demo}}

The Primary Write Client (PWC) Demonstration program, {\tt
dada\_pwc\_demo}, implements an example PWC interface.  It does not
actually acquire any data and therefore can be run on any computer.
This program has two modes of operation: free and locked.

In {\em free} mode, {\tt dada\_pwc\_demo} does not connect to the
Header and Data Blocks; therefore, it is not necessary to create the
shared memory resources and run a Primary Read Client program.  This
mode is most useful when testing the command interface and state
machine of the Primary Write Client.  To run in {\em free} mode,
simply type
\begin{verbatim}
dada_pwc_demo
\end{verbatim}

In {\em locked} mode, {\tt dada\_pwc\_demo} connects to the Header and
Data Blocks; therefore, it is necessary to first create the shared
memory resources and also run a Primary Read Client program, such as
{\tt dada\_dbdisk}.  This mode is most useful when testing the
interface between Primary Write Client, Header and Data Blocks, and
Primary Read Client software.  To run in {\em locked} mode, for
example
\begin{verbatim}
dada_db -d         # destroy existing shared memory resources
dada_db            # create new shared memory resources
dada_pwc_demo -l   # run in locked mode
\end{verbatim}
The first step is particularly useful when debugging.
In another window on the same machine, you might also run
\begin{verbatim}
dada_dbdisk -WD /tmp
\end{verbatim}

To connect with the PWC demonstration program and begin issuing
commands, simply run
\begin{verbatim}
telnet localhost 56026
\end{verbatim}
or replace {\tt localhost} with the name of the machine on which the
program is running.

\subsection{Primary Write Client Command, {\tt dada\_pwc\_command}}

It is also possible to control one or more instance of {\tt
dada\_pwc\_demo} using the Primary Write Client Command program, {\tt
dada\_pwc\_command}.  By default, {\tt dada\_pwc\_command} uses the
same port number (56026) as {\tt dada\_pwc\_demo}.  Therefore, if you
are running both programs on the same machine, it will be necessary to
specify a different port number for the command interface; e.g.
\begin{verbatim}
dada_pwc_command -p 20013
\end{verbatim}

The PWC command program must be configured as described in ???;
it also requires the use of specification files to prepare for recording.
